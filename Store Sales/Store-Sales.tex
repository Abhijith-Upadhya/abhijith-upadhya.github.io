% Options for packages loaded elsewhere
\PassOptionsToPackage{unicode}{hyperref}
\PassOptionsToPackage{hyphens}{url}
%
\documentclass[
]{article}
\usepackage{amsmath,amssymb}
\usepackage{iftex}
\ifPDFTeX
  \usepackage[T1]{fontenc}
  \usepackage[utf8]{inputenc}
  \usepackage{textcomp} % provide euro and other symbols
\else % if luatex or xetex
  \usepackage{unicode-math} % this also loads fontspec
  \defaultfontfeatures{Scale=MatchLowercase}
  \defaultfontfeatures[\rmfamily]{Ligatures=TeX,Scale=1}
\fi
\usepackage{lmodern}
\ifPDFTeX\else
  % xetex/luatex font selection
\fi
% Use upquote if available, for straight quotes in verbatim environments
\IfFileExists{upquote.sty}{\usepackage{upquote}}{}
\IfFileExists{microtype.sty}{% use microtype if available
  \usepackage[]{microtype}
  \UseMicrotypeSet[protrusion]{basicmath} % disable protrusion for tt fonts
}{}
\makeatletter
\@ifundefined{KOMAClassName}{% if non-KOMA class
  \IfFileExists{parskip.sty}{%
    \usepackage{parskip}
  }{% else
    \setlength{\parindent}{0pt}
    \setlength{\parskip}{6pt plus 2pt minus 1pt}}
}{% if KOMA class
  \KOMAoptions{parskip=half}}
\makeatother
\usepackage{xcolor}
\usepackage[margin=1in]{geometry}
\usepackage{color}
\usepackage{fancyvrb}
\newcommand{\VerbBar}{|}
\newcommand{\VERB}{\Verb[commandchars=\\\{\}]}
\DefineVerbatimEnvironment{Highlighting}{Verbatim}{commandchars=\\\{\}}
% Add ',fontsize=\small' for more characters per line
\usepackage{framed}
\definecolor{shadecolor}{RGB}{248,248,248}
\newenvironment{Shaded}{\begin{snugshade}}{\end{snugshade}}
\newcommand{\AlertTok}[1]{\textcolor[rgb]{0.94,0.16,0.16}{#1}}
\newcommand{\AnnotationTok}[1]{\textcolor[rgb]{0.56,0.35,0.01}{\textbf{\textit{#1}}}}
\newcommand{\AttributeTok}[1]{\textcolor[rgb]{0.13,0.29,0.53}{#1}}
\newcommand{\BaseNTok}[1]{\textcolor[rgb]{0.00,0.00,0.81}{#1}}
\newcommand{\BuiltInTok}[1]{#1}
\newcommand{\CharTok}[1]{\textcolor[rgb]{0.31,0.60,0.02}{#1}}
\newcommand{\CommentTok}[1]{\textcolor[rgb]{0.56,0.35,0.01}{\textit{#1}}}
\newcommand{\CommentVarTok}[1]{\textcolor[rgb]{0.56,0.35,0.01}{\textbf{\textit{#1}}}}
\newcommand{\ConstantTok}[1]{\textcolor[rgb]{0.56,0.35,0.01}{#1}}
\newcommand{\ControlFlowTok}[1]{\textcolor[rgb]{0.13,0.29,0.53}{\textbf{#1}}}
\newcommand{\DataTypeTok}[1]{\textcolor[rgb]{0.13,0.29,0.53}{#1}}
\newcommand{\DecValTok}[1]{\textcolor[rgb]{0.00,0.00,0.81}{#1}}
\newcommand{\DocumentationTok}[1]{\textcolor[rgb]{0.56,0.35,0.01}{\textbf{\textit{#1}}}}
\newcommand{\ErrorTok}[1]{\textcolor[rgb]{0.64,0.00,0.00}{\textbf{#1}}}
\newcommand{\ExtensionTok}[1]{#1}
\newcommand{\FloatTok}[1]{\textcolor[rgb]{0.00,0.00,0.81}{#1}}
\newcommand{\FunctionTok}[1]{\textcolor[rgb]{0.13,0.29,0.53}{\textbf{#1}}}
\newcommand{\ImportTok}[1]{#1}
\newcommand{\InformationTok}[1]{\textcolor[rgb]{0.56,0.35,0.01}{\textbf{\textit{#1}}}}
\newcommand{\KeywordTok}[1]{\textcolor[rgb]{0.13,0.29,0.53}{\textbf{#1}}}
\newcommand{\NormalTok}[1]{#1}
\newcommand{\OperatorTok}[1]{\textcolor[rgb]{0.81,0.36,0.00}{\textbf{#1}}}
\newcommand{\OtherTok}[1]{\textcolor[rgb]{0.56,0.35,0.01}{#1}}
\newcommand{\PreprocessorTok}[1]{\textcolor[rgb]{0.56,0.35,0.01}{\textit{#1}}}
\newcommand{\RegionMarkerTok}[1]{#1}
\newcommand{\SpecialCharTok}[1]{\textcolor[rgb]{0.81,0.36,0.00}{\textbf{#1}}}
\newcommand{\SpecialStringTok}[1]{\textcolor[rgb]{0.31,0.60,0.02}{#1}}
\newcommand{\StringTok}[1]{\textcolor[rgb]{0.31,0.60,0.02}{#1}}
\newcommand{\VariableTok}[1]{\textcolor[rgb]{0.00,0.00,0.00}{#1}}
\newcommand{\VerbatimStringTok}[1]{\textcolor[rgb]{0.31,0.60,0.02}{#1}}
\newcommand{\WarningTok}[1]{\textcolor[rgb]{0.56,0.35,0.01}{\textbf{\textit{#1}}}}
\usepackage{graphicx}
\makeatletter
\def\maxwidth{\ifdim\Gin@nat@width>\linewidth\linewidth\else\Gin@nat@width\fi}
\def\maxheight{\ifdim\Gin@nat@height>\textheight\textheight\else\Gin@nat@height\fi}
\makeatother
% Scale images if necessary, so that they will not overflow the page
% margins by default, and it is still possible to overwrite the defaults
% using explicit options in \includegraphics[width, height, ...]{}
\setkeys{Gin}{width=\maxwidth,height=\maxheight,keepaspectratio}
% Set default figure placement to htbp
\makeatletter
\def\fps@figure{htbp}
\makeatother
\setlength{\emergencystretch}{3em} % prevent overfull lines
\providecommand{\tightlist}{%
  \setlength{\itemsep}{0pt}\setlength{\parskip}{0pt}}
\setcounter{secnumdepth}{-\maxdimen} % remove section numbering
\ifLuaTeX
  \usepackage{selnolig}  % disable illegal ligatures
\fi
\IfFileExists{bookmark.sty}{\usepackage{bookmark}}{\usepackage{hyperref}}
\IfFileExists{xurl.sty}{\usepackage{xurl}}{} % add URL line breaks if available
\urlstyle{same}
\hypersetup{
  pdftitle={Store Sales},
  pdfauthor={Abhijith},
  hidelinks,
  pdfcreator={LaTeX via pandoc}}

\title{Store Sales}
\author{Abhijith}
\date{2024-10-20}

\begin{document}
\maketitle

\hypertarget{this-is-an-r-markdown-document-focusing-on-data-visualization}{%
\section{This is an R Markdown document focusing on Data
Visualization}\label{this-is-an-r-markdown-document-focusing-on-data-visualization}}

\begin{Shaded}
\begin{Highlighting}[]
\FunctionTok{library}\NormalTok{(RColorBrewer)}
\end{Highlighting}
\end{Shaded}

\begin{verbatim}
## Warning: package 'RColorBrewer' was built under R version 4.3.1
\end{verbatim}

\begin{Shaded}
\begin{Highlighting}[]
\FunctionTok{library}\NormalTok{(dplyr)}
\end{Highlighting}
\end{Shaded}

\begin{verbatim}
## Warning: package 'dplyr' was built under R version 4.3.3
\end{verbatim}

\begin{verbatim}
## 
## Attaching package: 'dplyr'
\end{verbatim}

\begin{verbatim}
## The following objects are masked from 'package:stats':
## 
##     filter, lag
\end{verbatim}

\begin{verbatim}
## The following objects are masked from 'package:base':
## 
##     intersect, setdiff, setequal, union
\end{verbatim}

\begin{Shaded}
\begin{Highlighting}[]
\FunctionTok{library}\NormalTok{(ggplot2)}
\end{Highlighting}
\end{Shaded}

\begin{verbatim}
## Warning: package 'ggplot2' was built under R version 4.3.3
\end{verbatim}

\begin{Shaded}
\begin{Highlighting}[]
\FunctionTok{library}\NormalTok{(tidyr)}
\end{Highlighting}
\end{Shaded}

\begin{verbatim}
## Warning: package 'tidyr' was built under R version 4.3.3
\end{verbatim}

\begin{Shaded}
\begin{Highlighting}[]
\FunctionTok{library}\NormalTok{(readr)}
\FunctionTok{library}\NormalTok{(forcats)}
\FunctionTok{library}\NormalTok{(maps)}
\end{Highlighting}
\end{Shaded}

\begin{verbatim}
## Warning: package 'maps' was built under R version 4.3.3
\end{verbatim}

\begin{Shaded}
\begin{Highlighting}[]
\FunctionTok{library}\NormalTok{(sf)}
\end{Highlighting}
\end{Shaded}

\begin{verbatim}
## Warning: package 'sf' was built under R version 4.3.1
\end{verbatim}

\begin{verbatim}
## Linking to GEOS 3.11.2, GDAL 3.6.2, PROJ 9.2.0; sf_use_s2() is TRUE
\end{verbatim}

\hypertarget{loading-the-dataset}{%
\subsection{Loading the dataset}\label{loading-the-dataset}}

\begin{Shaded}
\begin{Highlighting}[]
\CommentTok{\# Load the dataset}
\NormalTok{data }\OtherTok{\textless{}{-}} \FunctionTok{read\_csv}\NormalTok{(}\StringTok{"SuperStore\_Sales\_DataSet.csv"}\NormalTok{)}
\end{Highlighting}
\end{Shaded}

\begin{verbatim}
## Rows: 5901 Columns: 20
## -- Column specification --------------------------------------------------------
## Delimiter: ","
## chr (17): Order_ID, Order_Date, Ship_Date, Ship_Mode, Customer_ID, Customer_...
## dbl  (3): Sales, Quantity, Profit
## 
## i Use `spec()` to retrieve the full column specification for this data.
## i Specify the column types or set `show_col_types = FALSE` to quiet this message.
\end{verbatim}

\hypertarget{transforming-the-data-and-calculating-basic-metrics}{%
\subsection{Transforming the data and calculating basic
metrics}\label{transforming-the-data-and-calculating-basic-metrics}}

\begin{Shaded}
\begin{Highlighting}[]
\NormalTok{data }\OtherTok{\textless{}{-}}\NormalTok{ data }\SpecialCharTok{\%\textgreater{}\%}
  \FunctionTok{mutate}\NormalTok{(}\AttributeTok{Order\_Date =} \FunctionTok{as.Date}\NormalTok{(Order\_Date, }\AttributeTok{format=}\StringTok{"\%m/\%d/\%Y"}\NormalTok{),}
         \AttributeTok{Ship\_Date =} \FunctionTok{as.Date}\NormalTok{(Ship\_Date, }\AttributeTok{format=}\StringTok{"\%m/\%d/\%Y"}\NormalTok{),}
         \AttributeTok{Sales =} \FunctionTok{as.numeric}\NormalTok{(Sales),}
         \AttributeTok{Quantity =} \FunctionTok{as.numeric}\NormalTok{(Quantity),}
         \AttributeTok{Profit =} \FunctionTok{as.numeric}\NormalTok{(Profit)) }\SpecialCharTok{\%\textgreater{}\%}
  \FunctionTok{filter}\NormalTok{(}\SpecialCharTok{!}\FunctionTok{is.na}\NormalTok{(Sales), }\SpecialCharTok{!}\FunctionTok{is.na}\NormalTok{(Quantity), }\SpecialCharTok{!}\FunctionTok{is.na}\NormalTok{(Profit))}

\CommentTok{\# A. Calculate metrics}
\NormalTok{total\_orders }\OtherTok{\textless{}{-}} \FunctionTok{nrow}\NormalTok{(data)}
\NormalTok{total\_sales\_inr }\OtherTok{\textless{}{-}} \FunctionTok{sum}\NormalTok{(data}\SpecialCharTok{$}\NormalTok{Sales) }\SpecialCharTok{*} \FloatTok{84.10}  \CommentTok{\# Assuming 1 USD = 75 INR}
\NormalTok{average\_product\_quantity }\OtherTok{\textless{}{-}} \FunctionTok{mean}\NormalTok{(data}\SpecialCharTok{$}\NormalTok{Quantity)}
\NormalTok{average\_delivery\_days }\OtherTok{\textless{}{-}} \FunctionTok{mean}\NormalTok{(}\FunctionTok{as.numeric}\NormalTok{(data}\SpecialCharTok{$}\NormalTok{Ship\_Date }\SpecialCharTok{{-}}\NormalTok{ data}\SpecialCharTok{$}\NormalTok{Order\_Date))}

\CommentTok{\# Print results}
\FunctionTok{cat}\NormalTok{(}\StringTok{"Total Orders:"}\NormalTok{, total\_orders, }\StringTok{"}\SpecialCharTok{\textbackslash{}n}\StringTok{"}\NormalTok{)}
\end{Highlighting}
\end{Shaded}

\begin{verbatim}
## Total Orders: 5901
\end{verbatim}

\begin{Shaded}
\begin{Highlighting}[]
\FunctionTok{cat}\NormalTok{(}\StringTok{"Total Sales in Indian Rupees:"}\NormalTok{, total\_sales\_inr, }\StringTok{"}\SpecialCharTok{\textbackslash{}n}\StringTok{"}\NormalTok{)}
\end{Highlighting}
\end{Shaded}

\begin{verbatim}
## Total Sales in Indian Rupees: 131684144
\end{verbatim}

\begin{Shaded}
\begin{Highlighting}[]
\FunctionTok{cat}\NormalTok{(}\StringTok{"Average Product Quantity:"}\NormalTok{, average\_product\_quantity, }\StringTok{"}\SpecialCharTok{\textbackslash{}n}\StringTok{"}\NormalTok{)}
\end{Highlighting}
\end{Shaded}

\begin{verbatim}
## Average Product Quantity: 3.781901
\end{verbatim}

\begin{Shaded}
\begin{Highlighting}[]
\FunctionTok{cat}\NormalTok{(}\StringTok{"Average Delivery Days:"}\NormalTok{, average\_delivery\_days, }\StringTok{"}\SpecialCharTok{\textbackslash{}n}\StringTok{"}\NormalTok{)}
\end{Highlighting}
\end{Shaded}

\begin{verbatim}
## Average Delivery Days: NA
\end{verbatim}

\hypertarget{sales-and-profit-for-the-year-2019-and-2020}{%
\subsection{Sales and Profit for the year 2019 and
2020}\label{sales-and-profit-for-the-year-2019-and-2020}}

\begin{Shaded}
\begin{Highlighting}[]
\CommentTok{\# B. Create plots for Sales and Profit for 2019 and 2020}
\NormalTok{data\_years }\OtherTok{\textless{}{-}}\NormalTok{ data }\SpecialCharTok{\%\textgreater{}\%}
  \FunctionTok{filter}\NormalTok{(}\FunctionTok{format}\NormalTok{(Order\_Date, }\StringTok{"\%Y"}\NormalTok{) }\SpecialCharTok{\%in\%} \FunctionTok{c}\NormalTok{(}\StringTok{"2019"}\NormalTok{, }\StringTok{"2020"}\NormalTok{)) }\SpecialCharTok{\%\textgreater{}\%}
  \FunctionTok{group\_by}\NormalTok{(}\AttributeTok{year =} \FunctionTok{format}\NormalTok{(Order\_Date, }\StringTok{"\%Y"}\NormalTok{)) }\SpecialCharTok{\%\textgreater{}\%}
  \FunctionTok{summarise}\NormalTok{(}\AttributeTok{Total\_Sales =} \FunctionTok{sum}\NormalTok{(Sales), }\AttributeTok{Total\_Profit =} \FunctionTok{sum}\NormalTok{(Profit))}


\FunctionTok{ggplot}\NormalTok{(data\_years, }\FunctionTok{aes}\NormalTok{(}\AttributeTok{x =}\NormalTok{ year)) }\SpecialCharTok{+}
  \FunctionTok{geom\_bar}\NormalTok{(}\FunctionTok{aes}\NormalTok{(}\AttributeTok{y =}\NormalTok{ Total\_Sales), }\AttributeTok{stat =} \StringTok{"identity"}\NormalTok{, }\AttributeTok{fill =} \FunctionTok{brewer.pal}\NormalTok{(}\DecValTok{3}\NormalTok{, }\StringTok{"Blues"}\NormalTok{)[}\DecValTok{2}\NormalTok{], }\AttributeTok{width =} \FloatTok{0.6}\NormalTok{) }\SpecialCharTok{+}
  \FunctionTok{geom\_bar}\NormalTok{(}\FunctionTok{aes}\NormalTok{(}\AttributeTok{y =}\NormalTok{ Total\_Profit), }\AttributeTok{stat =} \StringTok{"identity"}\NormalTok{, }\AttributeTok{fill =} \FunctionTok{brewer.pal}\NormalTok{(}\DecValTok{3}\NormalTok{, }\StringTok{"Reds"}\NormalTok{)[}\DecValTok{2}\NormalTok{], }\AttributeTok{width =} \FloatTok{0.4}\NormalTok{) }\SpecialCharTok{+}
  \FunctionTok{labs}\NormalTok{(}\AttributeTok{title =} \StringTok{"Sales and Profit for 2019 and 2020"}\NormalTok{, }
       \AttributeTok{y =} \StringTok{"Amount"}\NormalTok{, }\AttributeTok{x =} \StringTok{"Year"}\NormalTok{) }\SpecialCharTok{+}
  \FunctionTok{theme\_minimal}\NormalTok{(}\AttributeTok{base\_size =} \DecValTok{15}\NormalTok{) }\SpecialCharTok{+} 
  \FunctionTok{theme}\NormalTok{(}\AttributeTok{legend.position =} \StringTok{"right"}\NormalTok{,}
        \AttributeTok{legend.title =} \FunctionTok{element\_blank}\NormalTok{(),}
        \AttributeTok{text =} \FunctionTok{element\_text}\NormalTok{(}\AttributeTok{family =} \StringTok{"serif"}\NormalTok{),}
        \AttributeTok{plot.title =} \FunctionTok{element\_text}\NormalTok{(}\AttributeTok{hjust =} \FloatTok{0.5}\NormalTok{, }\AttributeTok{size =} \DecValTok{16}\NormalTok{, }\AttributeTok{face =} \StringTok{"bold"}\NormalTok{))}
\end{Highlighting}
\end{Shaded}

\includegraphics{Store-Sales_files/figure-latex/sales2019/20-1.pdf}

\begin{Shaded}
\begin{Highlighting}[]
  \FunctionTok{scale\_y\_continuous}\NormalTok{(}\AttributeTok{labels =}\NormalTok{ scales}\SpecialCharTok{::}\NormalTok{comma)}
\end{Highlighting}
\end{Shaded}

\begin{verbatim}
## <ScaleContinuousPosition>
##  Range:  
##  Limits:    0 --    1
\end{verbatim}

\hypertarget{category-details}{%
\subsection{Category Details}\label{category-details}}

\begin{Shaded}
\begin{Highlighting}[]
\CommentTok{\# C. Create donut charts for various categories}
\NormalTok{sales\_by\_region }\OtherTok{\textless{}{-}}\NormalTok{ data }\SpecialCharTok{\%\textgreater{}\%}
  \FunctionTok{group\_by}\NormalTok{(Region) }\SpecialCharTok{\%\textgreater{}\%}
  \FunctionTok{summarise}\NormalTok{(}\AttributeTok{Sales =} \FunctionTok{sum}\NormalTok{(Sales))}

\NormalTok{sales\_by\_payment\_mode }\OtherTok{\textless{}{-}}\NormalTok{ data }\SpecialCharTok{\%\textgreater{}\%}
  \FunctionTok{group\_by}\NormalTok{(Payment\_Mode) }\SpecialCharTok{\%\textgreater{}\%}
  \FunctionTok{summarise}\NormalTok{(}\AttributeTok{Sales =} \FunctionTok{sum}\NormalTok{(Sales))}

\NormalTok{sales\_by\_segment }\OtherTok{\textless{}{-}}\NormalTok{ data }\SpecialCharTok{\%\textgreater{}\%}
  \FunctionTok{group\_by}\NormalTok{(Segment) }\SpecialCharTok{\%\textgreater{}\%}
  \FunctionTok{summarise}\NormalTok{(}\AttributeTok{Sales =} \FunctionTok{sum}\NormalTok{(Sales))}

\NormalTok{sales\_by\_shipment\_mode }\OtherTok{\textless{}{-}}\NormalTok{ data }\SpecialCharTok{\%\textgreater{}\%}
  \FunctionTok{group\_by}\NormalTok{(Ship\_Mode) }\SpecialCharTok{\%\textgreater{}\%}
  \FunctionTok{summarise}\NormalTok{(}\AttributeTok{Sales =} \FunctionTok{sum}\NormalTok{(Sales))}

\NormalTok{sales\_by\_category }\OtherTok{\textless{}{-}}\NormalTok{ data }\SpecialCharTok{\%\textgreater{}\%}
  \FunctionTok{group\_by}\NormalTok{(Category) }\SpecialCharTok{\%\textgreater{}\%}
  \FunctionTok{summarise}\NormalTok{(}\AttributeTok{Sales =} \FunctionTok{sum}\NormalTok{(Sales))}

\NormalTok{sales\_by\_sub\_category }\OtherTok{\textless{}{-}}\NormalTok{ data }\SpecialCharTok{\%\textgreater{}\%}
  \FunctionTok{group\_by}\NormalTok{(Sub\_Category) }\SpecialCharTok{\%\textgreater{}\%}
  \FunctionTok{summarise}\NormalTok{(}\AttributeTok{Sales =} \FunctionTok{sum}\NormalTok{(Sales))}


\CommentTok{\# Function to create donut chart}
\NormalTok{create\_donut\_chart }\OtherTok{\textless{}{-}} \ControlFlowTok{function}\NormalTok{(data, category, title) \{}
  \FunctionTok{ggplot}\NormalTok{(data, }\FunctionTok{aes}\NormalTok{(}\AttributeTok{x =} \DecValTok{2}\NormalTok{, }\AttributeTok{y =}\NormalTok{ Sales, }\AttributeTok{fill =} \SpecialCharTok{!!}\FunctionTok{sym}\NormalTok{(category))) }\SpecialCharTok{+} 
    \FunctionTok{geom\_col}\NormalTok{(}\AttributeTok{width =} \DecValTok{1}\NormalTok{, }\AttributeTok{color =} \StringTok{"white"}\NormalTok{) }\SpecialCharTok{+} 
    \FunctionTok{coord\_polar}\NormalTok{(}\AttributeTok{theta =} \StringTok{"y"}\NormalTok{) }\SpecialCharTok{+}
    \FunctionTok{xlim}\NormalTok{(}\FloatTok{0.5}\NormalTok{, }\FloatTok{2.5}\NormalTok{) }\SpecialCharTok{+} \CommentTok{\# This makes it a donut chart instead of a pie chart}
    \FunctionTok{theme\_void}\NormalTok{() }\SpecialCharTok{+} 
    \FunctionTok{theme}\NormalTok{(}\AttributeTok{legend.position =} \StringTok{"right"}\NormalTok{,}
          \AttributeTok{legend.title =} \FunctionTok{element\_blank}\NormalTok{(),}
          \AttributeTok{text =} \FunctionTok{element\_text}\NormalTok{(}\AttributeTok{family =} \StringTok{"serif"}\NormalTok{),}
          \AttributeTok{plot.title =} \FunctionTok{element\_text}\NormalTok{(}\AttributeTok{hjust =} \FloatTok{0.5}\NormalTok{, }\AttributeTok{size =} \DecValTok{16}\NormalTok{, }\AttributeTok{face =} \StringTok{"bold"}\NormalTok{)) }\SpecialCharTok{+}
    \FunctionTok{labs}\NormalTok{(}\AttributeTok{title =}\NormalTok{ title, }\AttributeTok{fill =}\NormalTok{ category)}
\NormalTok{\}}
\end{Highlighting}
\end{Shaded}

\hypertarget{plotting-the-necessary-details}{%
\subsection{Plotting the necessary
details}\label{plotting-the-necessary-details}}

\hypertarget{a.-sales-by-region}{%
\section{a. Sales by Region}\label{a.-sales-by-region}}

This chart shows the distribution of total sales across different
regions. Regions with larger portions contribute more to the company's
revenue. This visualization helps in identifying the most lucrative
regions for focusing marketing campaigns or expansion strategies.

By understanding which region contributes the most to sales, the company
can focus on region-specific promotions or address underperforming areas
with targeted strategies.

\begin{Shaded}
\begin{Highlighting}[]
\FunctionTok{create\_donut\_chart}\NormalTok{(sales\_by\_region, }\StringTok{"Region"}\NormalTok{, }\StringTok{"Sales by Region"}\NormalTok{)}
\end{Highlighting}
\end{Shaded}

\includegraphics{Store-Sales_files/figure-latex/region-1.pdf}

\hypertarget{b.-sales-by-payment-mode}{%
\section{b. Sales by Payment Mode}\label{b.-sales-by-payment-mode}}

This chart highlights the popularity of different payment methods. A
higher portion for a specific payment mode means customers prefer that
method.

Recognizing the preferred payment modes allows the business to ensure
that these options are always available and improve the payment
experience for users, potentially increasing conversion rates.

\begin{Shaded}
\begin{Highlighting}[]
\FunctionTok{create\_donut\_chart}\NormalTok{(sales\_by\_payment\_mode, }\StringTok{"Payment\_Mode"}\NormalTok{, }\StringTok{"Sales by Payment Mode"}\NormalTok{)}
\end{Highlighting}
\end{Shaded}

\includegraphics{Store-Sales_files/figure-latex/payment_mode-1.pdf}

\hypertarget{c.-sales-by-segment}{%
\section{c.~Sales by Segment}\label{c.-sales-by-segment}}

This shows how much revenue is generated from various customer segments,
such as ``Consumer,'' ``Corporate,'' and ``Home Office.''

Identifying the customer segments generating the most revenue helps in
tailoring marketing campaigns to target high-value customers and
maintain growth.

\begin{Shaded}
\begin{Highlighting}[]
\FunctionTok{create\_donut\_chart}\NormalTok{(sales\_by\_segment, }\StringTok{"Segment"}\NormalTok{, }\StringTok{"Sales by Segment"}\NormalTok{)}
\end{Highlighting}
\end{Shaded}

\includegraphics{Store-Sales_files/figure-latex/segments-1.pdf}

\hypertarget{d.-sales-by-shipment-mode}{%
\section{d.~Sales by Shipment Mode}\label{d.-sales-by-shipment-mode}}

This chart shows how different shipping methods contribute to total
sales. Shipment options could range from standard to express delivery.

Understanding which shipping methods are preferred by customers enables
optimization of delivery services, improving customer satisfaction and
reducing logistics costs.

\begin{Shaded}
\begin{Highlighting}[]
\FunctionTok{create\_donut\_chart}\NormalTok{(sales\_by\_shipment\_mode, }\StringTok{"Ship\_Mode"}\NormalTok{, }\StringTok{"Sales by Shipment Mode"}\NormalTok{)}
\end{Highlighting}
\end{Shaded}

\includegraphics{Store-Sales_files/figure-latex/shipment_mode-1.pdf}

\hypertarget{e.-sales-by-category-and-sub-category}{%
\section{e. Sales by Category and
Sub-Category}\label{e.-sales-by-category-and-sub-category}}

This visualizes the distribution of sales among various product
categories and sub-categories, such as technology, furniture, or office
supplies.

Identifying the top-performing product categories helps in inventory
planning, improving product offerings, and expanding product lines that
resonate with customers.

\begin{Shaded}
\begin{Highlighting}[]
\FunctionTok{create\_donut\_chart}\NormalTok{(sales\_by\_category, }\StringTok{"Category"}\NormalTok{, }\StringTok{"Sales by Category"}\NormalTok{)}
\end{Highlighting}
\end{Shaded}

\includegraphics{Store-Sales_files/figure-latex/category-1.pdf}

\begin{Shaded}
\begin{Highlighting}[]
\FunctionTok{create\_donut\_chart}\NormalTok{(sales\_by\_sub\_category, }\StringTok{"Sub\_Category"}\NormalTok{, }\StringTok{"Sales by Sub{-}Category"}\NormalTok{)}
\end{Highlighting}
\end{Shaded}

\includegraphics{Store-Sales_files/figure-latex/category-2.pdf}

\hypertarget{observations}{%
\subsection{Observations}\label{observations}}

\begin{Shaded}
\begin{Highlighting}[]
\CommentTok{\# D. Create stacked charts for various categories}
\NormalTok{popular\_payment\_modes }\OtherTok{\textless{}{-}}\NormalTok{ data }\SpecialCharTok{\%\textgreater{}\%}
  \FunctionTok{group\_by}\NormalTok{(Payment\_Mode) }\SpecialCharTok{\%\textgreater{}\%}
  \FunctionTok{summarise}\NormalTok{(}\AttributeTok{Total\_Sales =} \FunctionTok{sum}\NormalTok{(Sales), }\AttributeTok{Total\_Profit =} \FunctionTok{sum}\NormalTok{(Profit))}

\NormalTok{product\_segments }\OtherTok{\textless{}{-}}\NormalTok{ data }\SpecialCharTok{\%\textgreater{}\%}
  \FunctionTok{group\_by}\NormalTok{(Segment) }\SpecialCharTok{\%\textgreater{}\%}
  \FunctionTok{summarise}\NormalTok{(}\AttributeTok{Total\_Sales =} \FunctionTok{sum}\NormalTok{(Sales))}

\NormalTok{categories\_data }\OtherTok{\textless{}{-}}\NormalTok{ data }\SpecialCharTok{\%\textgreater{}\%}
  \FunctionTok{group\_by}\NormalTok{(Category) }\SpecialCharTok{\%\textgreater{}\%}
  \FunctionTok{summarise}\NormalTok{(}\AttributeTok{Total\_Sales =} \FunctionTok{sum}\NormalTok{(Sales))}

\NormalTok{delivery\_type\_region }\OtherTok{\textless{}{-}}\NormalTok{ data }\SpecialCharTok{\%\textgreater{}\%}
  \FunctionTok{group\_by}\NormalTok{(Ship\_Mode, Region) }\SpecialCharTok{\%\textgreater{}\%}
  \FunctionTok{summarise}\NormalTok{(}\AttributeTok{Total\_Sales =} \FunctionTok{sum}\NormalTok{(Sales))}
\end{Highlighting}
\end{Shaded}

\begin{verbatim}
## `summarise()` has grouped output by 'Ship_Mode'. You can override using the
## `.groups` argument.
\end{verbatim}

\begin{Shaded}
\begin{Highlighting}[]
\CommentTok{\# Enhanced Stacked Bar Chart}
\NormalTok{create\_stacked\_bar\_chart }\OtherTok{\textless{}{-}} \ControlFlowTok{function}\NormalTok{(data, x\_var, fill\_var, title) \{}
  \FunctionTok{ggplot}\NormalTok{(data, }\FunctionTok{aes\_string}\NormalTok{(}\AttributeTok{x =}\NormalTok{ x\_var, }\AttributeTok{y =} \StringTok{"Total\_Sales"}\NormalTok{, }\AttributeTok{fill =}\NormalTok{ fill\_var)) }\SpecialCharTok{+} 
    \FunctionTok{geom\_bar}\NormalTok{(}\AttributeTok{stat =} \StringTok{"identity"}\NormalTok{, }\AttributeTok{color =} \StringTok{"black"}\NormalTok{, }\AttributeTok{width =} \FloatTok{0.6}\NormalTok{) }\SpecialCharTok{+} 
    \FunctionTok{labs}\NormalTok{(}\AttributeTok{title =}\NormalTok{ title, }\AttributeTok{x =}\NormalTok{ x\_var, }\AttributeTok{fill =}\NormalTok{ fill\_var) }\SpecialCharTok{+} 
    \FunctionTok{theme\_minimal}\NormalTok{(}\AttributeTok{base\_size =} \DecValTok{14}\NormalTok{) }\SpecialCharTok{+}
    \FunctionTok{theme}\NormalTok{(}\AttributeTok{text =} \FunctionTok{element\_text}\NormalTok{(}\AttributeTok{family =} \StringTok{"serif"}\NormalTok{),}
          \AttributeTok{plot.title =} \FunctionTok{element\_text}\NormalTok{(}\AttributeTok{hjust =} \FloatTok{0.5}\NormalTok{, }\AttributeTok{face =} \StringTok{"bold"}\NormalTok{),}
          \AttributeTok{axis.title.y =} \FunctionTok{element\_text}\NormalTok{(}\AttributeTok{face =} \StringTok{"bold"}\NormalTok{)) }\SpecialCharTok{+}
    \FunctionTok{scale\_y\_continuous}\NormalTok{(}\AttributeTok{labels =}\NormalTok{ scales}\SpecialCharTok{::}\NormalTok{comma, }\AttributeTok{expand =} \FunctionTok{expansion}\NormalTok{(}\AttributeTok{mult =} \FunctionTok{c}\NormalTok{(}\DecValTok{0}\NormalTok{, }\FloatTok{0.1}\NormalTok{))) }\SpecialCharTok{+}
    \FunctionTok{scale\_fill\_brewer}\NormalTok{(}\AttributeTok{palette =} \StringTok{"Dark2"}\NormalTok{)}
\NormalTok{\}}
\end{Highlighting}
\end{Shaded}

\hypertarget{a.-most-popular-payment-modes}{%
\section{a. Most Popular Payment
Modes}\label{a.-most-popular-payment-modes}}

This chart illustrates the breakdown of total sales by different payment
modes. If some payment options significantly outperform others, it
suggests customer preferences for convenience or security in
transactions.

Businesses can focus on offering seamless experiences for popular
payment modes, while considering removing or rethinking less popular
options. Additionally, promoting underused methods with incentives might
increase usage.

\begin{Shaded}
\begin{Highlighting}[]
\FunctionTok{create\_stacked\_bar\_chart}\NormalTok{(popular\_payment\_modes, }\StringTok{"Payment\_Mode"}\NormalTok{, }\StringTok{"Payment\_Mode"}\NormalTok{, }\StringTok{"Most Popular Payment Modes"}\NormalTok{)}
\end{Highlighting}
\end{Shaded}

\begin{verbatim}
## Warning: `aes_string()` was deprecated in ggplot2 3.0.0.
## i Please use tidy evaluation idioms with `aes()`.
## i See also `vignette("ggplot2-in-packages")` for more information.
## This warning is displayed once every 8 hours.
## Call `lifecycle::last_lifecycle_warnings()` to see where this warning was
## generated.
\end{verbatim}

\includegraphics{Store-Sales_files/figure-latex/popular_payment-1.pdf}

\hypertarget{b.-product-segments}{%
\section{b. Product Segments}\label{b.-product-segments}}

This chart shows the sales performance across different product
segments. Some segments may dominate the sales, showing strong customer
demand.

The business should increase investment and promotions in segments with
higher sales while investigating why certain segments underperform. For
lower-performing segments, customer surveys or research might reveal how
to enhance their appeal.

\begin{Shaded}
\begin{Highlighting}[]
\FunctionTok{create\_stacked\_bar\_chart}\NormalTok{(product\_segments, }\StringTok{"Segment"}\NormalTok{, }\StringTok{"Segment"}\NormalTok{, }\StringTok{"Product Segments"}\NormalTok{)}
\end{Highlighting}
\end{Shaded}

\includegraphics{Store-Sales_files/figure-latex/segment-1.pdf}

\hypertarget{c.-categories}{%
\section{c.~Categories}\label{c.-categories}}

This stacked chart illustrates the sales contribution of various product
categories, helping in the identification of high and low performers.

The business can prioritize marketing and product development efforts in
the most profitable categories. For the lower-selling categories, it
might indicate a need for product line refreshes or discontinuation.

\begin{Shaded}
\begin{Highlighting}[]
\FunctionTok{create\_stacked\_bar\_chart}\NormalTok{(categories\_data, }\StringTok{"Category"}\NormalTok{, }\StringTok{"Category"}\NormalTok{, }\StringTok{"Categories"}\NormalTok{)}
\end{Highlighting}
\end{Shaded}

\includegraphics{Store-Sales_files/figure-latex/categories-1.pdf}

\hypertarget{d.-delivery-type-by-region-of-order}{%
\section{d.~Delivery Type by Region of
Order}\label{d.-delivery-type-by-region-of-order}}

This chart represents how different regions prefer shipment methods,
revealing geographic preferences.

The business can optimize its logistics strategies based on regional
preferences, such as offering faster shipping options in regions that
value speed or cheaper options where cost is more sensitive.

\begin{Shaded}
\begin{Highlighting}[]
\FunctionTok{create\_stacked\_bar\_chart}\NormalTok{(delivery\_type\_region, }\StringTok{"Ship\_Mode"}\NormalTok{, }\StringTok{"Region"}\NormalTok{, }\StringTok{"Delivery Type by Region of Order"}\NormalTok{)}
\end{Highlighting}
\end{Shaded}

\includegraphics{Store-Sales_files/figure-latex/delivery-1.pdf}

\hypertarget{steps-for-improvement}{%
\subsection{Steps for improvement}\label{steps-for-improvement}}

\begin{Shaded}
\begin{Highlighting}[]
\CommentTok{\# F. Create stacked chart for sales and profit analysis}
\NormalTok{top\_sales\_profit\_payment\_modes }\OtherTok{\textless{}{-}}\NormalTok{ data }\SpecialCharTok{\%\textgreater{}\%}
  \FunctionTok{group\_by}\NormalTok{(Payment\_Mode) }\SpecialCharTok{\%\textgreater{}\%}
  \FunctionTok{summarise}\NormalTok{(}\AttributeTok{Total\_Sales =} \FunctionTok{sum}\NormalTok{(Sales), }\AttributeTok{Total\_Profit =} \FunctionTok{sum}\NormalTok{(Profit))}

\NormalTok{customer\_segments\_data }\OtherTok{\textless{}{-}}\NormalTok{ data }\SpecialCharTok{\%\textgreater{}\%}
  \FunctionTok{group\_by}\NormalTok{(Segment) }\SpecialCharTok{\%\textgreater{}\%}
  \FunctionTok{summarise}\NormalTok{(}\AttributeTok{Total\_Sales =} \FunctionTok{sum}\NormalTok{(Sales), }\AttributeTok{Total\_Profit =} \FunctionTok{sum}\NormalTok{(Profit))}

\NormalTok{product\_categories\_data }\OtherTok{\textless{}{-}}\NormalTok{ data }\SpecialCharTok{\%\textgreater{}\%}
  \FunctionTok{group\_by}\NormalTok{(Category) }\SpecialCharTok{\%\textgreater{}\%}
  \FunctionTok{summarise}\NormalTok{(}\AttributeTok{Total\_Sales =} \FunctionTok{sum}\NormalTok{(Sales), }\AttributeTok{Total\_Profit =} \FunctionTok{sum}\NormalTok{(Profit))}

\NormalTok{shipping\_mode\_region\_data }\OtherTok{\textless{}{-}}\NormalTok{ data }\SpecialCharTok{\%\textgreater{}\%}
  \FunctionTok{group\_by}\NormalTok{(Ship\_Mode, Region) }\SpecialCharTok{\%\textgreater{}\%}
  \FunctionTok{summarise}\NormalTok{(}\AttributeTok{Total\_Sales =} \FunctionTok{sum}\NormalTok{(Sales), }\AttributeTok{Total\_Profit =} \FunctionTok{sum}\NormalTok{(Profit))}
\end{Highlighting}
\end{Shaded}

\begin{verbatim}
## `summarise()` has grouped output by 'Ship_Mode'. You can override using the
## `.groups` argument.
\end{verbatim}

\begin{Shaded}
\begin{Highlighting}[]
\NormalTok{create\_stacked\_bar\_chart1 }\OtherTok{\textless{}{-}} \ControlFlowTok{function}\NormalTok{(data, x\_var, fill\_var, title) \{}
  \FunctionTok{ggplot}\NormalTok{(data, }\FunctionTok{aes\_string}\NormalTok{(}\AttributeTok{x =}\NormalTok{ x\_var, }\AttributeTok{y =} \StringTok{"Total\_Sales"}\NormalTok{, }\AttributeTok{fill =}\NormalTok{ fill\_var)) }\SpecialCharTok{+} 
    \FunctionTok{geom\_bar}\NormalTok{(}\AttributeTok{stat =} \StringTok{"identity"}\NormalTok{, }\AttributeTok{color =} \StringTok{"black"}\NormalTok{, }\AttributeTok{width =} \FloatTok{0.6}\NormalTok{) }\SpecialCharTok{+} 
    \FunctionTok{labs}\NormalTok{(}\AttributeTok{title =}\NormalTok{ title, }\AttributeTok{x =}\NormalTok{ x\_var, }\AttributeTok{fill =}\NormalTok{ fill\_var) }\SpecialCharTok{+} 
    \FunctionTok{theme\_minimal}\NormalTok{(}\AttributeTok{base\_size =} \DecValTok{14}\NormalTok{) }\SpecialCharTok{+}
    \FunctionTok{theme}\NormalTok{(}\AttributeTok{text =} \FunctionTok{element\_text}\NormalTok{(}\AttributeTok{family =} \StringTok{"serif"}\NormalTok{),}
          \AttributeTok{plot.title =} \FunctionTok{element\_text}\NormalTok{(}\AttributeTok{hjust =} \FloatTok{0.5}\NormalTok{, }\AttributeTok{face =} \StringTok{"bold"}\NormalTok{),}
          \AttributeTok{axis.title.y =} \FunctionTok{element\_text}\NormalTok{(}\AttributeTok{face =} \StringTok{"bold"}\NormalTok{)) }\SpecialCharTok{+}
    \FunctionTok{scale\_y\_continuous}\NormalTok{(}\AttributeTok{labels =}\NormalTok{ scales}\SpecialCharTok{::}\NormalTok{comma, }\AttributeTok{expand =} \FunctionTok{expansion}\NormalTok{(}\AttributeTok{mult =} \FunctionTok{c}\NormalTok{(}\DecValTok{0}\NormalTok{, }\FloatTok{0.1}\NormalTok{))) }\SpecialCharTok{+}
    \FunctionTok{scale\_fill\_gradient}\NormalTok{(}\AttributeTok{low =} \StringTok{"lightblue"}\NormalTok{, }\AttributeTok{high =} \StringTok{"darkblue"}\NormalTok{)  }\CommentTok{\# Use gradient for continuous values}
\NormalTok{\}}
\end{Highlighting}
\end{Shaded}

\hypertarget{a.-top-sales-and-profit-by-payment-modes}{%
\section{a. Top Sales and Profit by Payment
Modes}\label{a.-top-sales-and-profit-by-payment-modes}}

Improving Sales: Introduce offers such as discounts or cashback for less
popular payment methods to encourage more widespread use. Marketing
Strategy: Highlight secure, easy-to-use payment modes in advertisements,
particularly emphasizing mobile-friendly and fast checkout options.
Cost-Cutting Steps: Reduce transaction fees by negotiating better rates
with payment processors, particularly for less common payment methods.
Improving Product Range: Align product promotions with specific payment
methods, offering product bundles or exclusive deals for payments made
through a specific mode.

\begin{Shaded}
\begin{Highlighting}[]
\FunctionTok{create\_stacked\_bar\_chart1}\NormalTok{(top\_sales\_profit\_payment\_modes, }\StringTok{"Payment\_Mode"}\NormalTok{, }\StringTok{"Total\_Sales"}\NormalTok{, }\StringTok{"Top Sales and Profit by Payment Modes"}\NormalTok{)}
\end{Highlighting}
\end{Shaded}

\includegraphics{Store-Sales_files/figure-latex/top_sales_profit-1.pdf}

\hypertarget{b.-customer-segments}{%
\section{b. Customer Segments}\label{b.-customer-segments}}

Improving Sales: Tailor personalized offers and loyalty programs based
on customer segments, especially for high-value consumers. Marketing
Strategy: Create segment-specific campaigns focusing on corporate
customers with bulk purchase incentives, and consumer segments with
seasonal discounts. Cost-Cutting Steps: Streamline marketing efforts for
underperforming segments, focusing more on profitable demographics.
Improving Product Range: Expand the range of products based on the
preferences of the top-performing customer segments, such as launching
premium lines for corporate clients.

\begin{Shaded}
\begin{Highlighting}[]
\FunctionTok{create\_stacked\_bar\_chart1}\NormalTok{(customer\_segments\_data, }\StringTok{"Segment"}\NormalTok{, }\StringTok{"Total\_Sales"}\NormalTok{, }\StringTok{"Customer\_Segments"}\NormalTok{)}
\end{Highlighting}
\end{Shaded}

\includegraphics{Store-Sales_files/figure-latex/total_segments-1.pdf}

\hypertarget{c.-product-categories}{%
\section{c.~Product Categories}\label{c.-product-categories}}

Improving Sales: Offer bundle deals or discounts on lower-performing
categories to stimulate interest and clear excess stock. Marketing
Strategy: Focus advertisements on high-performing categories, and create
promotional campaigns to boost awareness and demand for less popular
ones. Cost-Cutting Steps: Optimize inventory levels for categories with
declining sales, reducing excess stock and associated carrying costs.
Improving Product Range: Introduce complementary products or features to
high-selling categories to capitalize on existing demand and upsell.

\begin{Shaded}
\begin{Highlighting}[]
\FunctionTok{create\_stacked\_bar\_chart1}\NormalTok{(product\_categories\_data, }\StringTok{"Category"}\NormalTok{, }\StringTok{"Total\_Sales"}\NormalTok{, }\StringTok{"Product\_Categories"}\NormalTok{)}
\end{Highlighting}
\end{Shaded}

\includegraphics{Store-Sales_files/figure-latex/unnamed-chunk-1-1.pdf}

\hypertarget{d.-shipping-mode-by-regions}{%
\section{d.~Shipping Mode by
Regions}\label{d.-shipping-mode-by-regions}}

Improving Sales: Offer region-specific promotions, such as free shipping
or faster delivery in high-demand areas. Marketing Strategy: Emphasize
the advantages of different shipping options in marketing campaigns
(e.g., faster deliveries in urban areas). Cost-Cutting Steps: Implement
more efficient shipping logistics for regions that show a preference for
certain delivery modes, reducing costs associated with expedited
shipping. Improving Product Range: Consider offering region-specific
products based on shipping preferences, especially where shipping
logistics play a big role in customer satisfaction.

\begin{Shaded}
\begin{Highlighting}[]
\FunctionTok{create\_stacked\_bar\_chart}\NormalTok{(shipping\_mode\_region\_data, }\StringTok{"Ship\_Mode"}\NormalTok{, }\StringTok{"Region"}\NormalTok{, }\StringTok{"Shipping Mode by Regions of Order"}\NormalTok{)}
\end{Highlighting}
\end{Shaded}

\includegraphics{Store-Sales_files/figure-latex/unnamed-chunk-2-1.pdf}

\hypertarget{map-depicting-the-profit-generating-states}{%
\subsection{Map depicting the Profit generating
States}\label{map-depicting-the-profit-generating-states}}

\begin{Shaded}
\begin{Highlighting}[]
\NormalTok{profit\_by\_state }\OtherTok{\textless{}{-}}\NormalTok{ data }\SpecialCharTok{\%\textgreater{}\%}
  \FunctionTok{group\_by}\NormalTok{(State) }\SpecialCharTok{\%\textgreater{}\%}
  \FunctionTok{summarise}\NormalTok{(}\AttributeTok{Total\_Profit =} \FunctionTok{sum}\NormalTok{(Profit, }\AttributeTok{na.rm =} \ConstantTok{TRUE}\NormalTok{))}

\CommentTok{\# Get US state map data}
\NormalTok{us\_states }\OtherTok{\textless{}{-}} \FunctionTok{map\_data}\NormalTok{(}\StringTok{"state"}\NormalTok{)}

\CommentTok{\# Merge the state map data with profit data}
\CommentTok{\# Convert state names in both datasets to lowercase for matching}
\NormalTok{profit\_by\_state}\SpecialCharTok{$}\NormalTok{State }\OtherTok{\textless{}{-}} \FunctionTok{tolower}\NormalTok{(profit\_by\_state}\SpecialCharTok{$}\NormalTok{State)}
\NormalTok{us\_states}\SpecialCharTok{$}\NormalTok{region }\OtherTok{\textless{}{-}} \FunctionTok{tolower}\NormalTok{(us\_states}\SpecialCharTok{$}\NormalTok{region)}

\CommentTok{\# Merge map data with profit data}
\NormalTok{map\_data }\OtherTok{\textless{}{-}} \FunctionTok{merge}\NormalTok{(us\_states, profit\_by\_state, }\AttributeTok{by.x =} \StringTok{"region"}\NormalTok{, }\AttributeTok{by.y =} \StringTok{"State"}\NormalTok{, }\AttributeTok{all.x =} \ConstantTok{TRUE}\NormalTok{)}

\CommentTok{\# Create a map with profit levels}
\FunctionTok{ggplot}\NormalTok{(map\_data, }\FunctionTok{aes}\NormalTok{(}\AttributeTok{x =}\NormalTok{ long, }\AttributeTok{y =}\NormalTok{ lat, }\AttributeTok{group =}\NormalTok{ group, }\AttributeTok{fill =}\NormalTok{ Total\_Profit)) }\SpecialCharTok{+}
  \FunctionTok{geom\_polygon}\NormalTok{(}\AttributeTok{color =} \StringTok{"white"}\NormalTok{, }\AttributeTok{size =} \FloatTok{0.2}\NormalTok{) }\SpecialCharTok{+}  \CommentTok{\# Thinner borders for better clarity}
  \FunctionTok{scale\_fill\_gradient}\NormalTok{(}\AttributeTok{low =} \StringTok{"lightyellow"}\NormalTok{, }\AttributeTok{high =} \StringTok{"darkred"}\NormalTok{, }\AttributeTok{na.value =} \StringTok{"gray90"}\NormalTok{, }\AttributeTok{name =} \StringTok{"Profit"}\NormalTok{) }\SpecialCharTok{+} 
  \FunctionTok{coord\_fixed}\NormalTok{(}\FloatTok{1.3}\NormalTok{) }\SpecialCharTok{+} 
  \FunctionTok{labs}\NormalTok{(}\AttributeTok{title =} \StringTok{"Profit by States"}\NormalTok{) }\SpecialCharTok{+}
  \FunctionTok{theme\_void}\NormalTok{() }\SpecialCharTok{+}
  \FunctionTok{theme}\NormalTok{(}\AttributeTok{text =} \FunctionTok{element\_text}\NormalTok{(}\AttributeTok{family =} \StringTok{"serif"}\NormalTok{),}
        \AttributeTok{plot.title =} \FunctionTok{element\_text}\NormalTok{(}\AttributeTok{hjust =} \FloatTok{0.5}\NormalTok{, }\AttributeTok{size =} \DecValTok{18}\NormalTok{, }\AttributeTok{face =} \StringTok{"bold"}\NormalTok{)) }\SpecialCharTok{+}
  \FunctionTok{geom\_text}\NormalTok{(}\AttributeTok{data =}\NormalTok{ map\_data }\SpecialCharTok{\%\textgreater{}\%}
              \FunctionTok{group\_by}\NormalTok{(region) }\SpecialCharTok{\%\textgreater{}\%}
              \FunctionTok{summarise}\NormalTok{(}\AttributeTok{long =} \FunctionTok{mean}\NormalTok{(long), }\AttributeTok{lat =} \FunctionTok{mean}\NormalTok{(lat)),}
            \FunctionTok{aes}\NormalTok{(}\AttributeTok{label =}\NormalTok{ region, }\AttributeTok{x =}\NormalTok{ long, }\AttributeTok{y =}\NormalTok{ lat), }\AttributeTok{size =} \DecValTok{2}\NormalTok{, }\AttributeTok{color =} \StringTok{"black"}\NormalTok{, }\AttributeTok{inherit.aes =} \ConstantTok{FALSE}\NormalTok{)  }\CommentTok{\# Add state names}
\end{Highlighting}
\end{Shaded}

\begin{verbatim}
## Warning: Using `size` aesthetic for lines was deprecated in ggplot2 3.4.0.
## i Please use `linewidth` instead.
## This warning is displayed once every 8 hours.
## Call `lifecycle::last_lifecycle_warnings()` to see where this warning was
## generated.
\end{verbatim}

\includegraphics{Store-Sales_files/figure-latex/profit_map-1.pdf}

\hypertarget{ways-to-increase-customer-retention-and-customer-loyalty}{%
\subsection{Ways to increase Customer Retention and Customer
Loyalty}\label{ways-to-increase-customer-retention-and-customer-loyalty}}

Personalized Communication: Send personalized email campaigns or push
notifications based on customer preferences, previous orders, and
purchase frequency. Loyalty Programs: Implement a tiered loyalty program
where returning customers earn points for purchases, encouraging repeat
business. Offer rewards like discounts, exclusive access to new
products, or free shipping. Customer Feedback: Regularly collect and
analyze customer feedback, offering improvements based on this data to
ensure a more personalized shopping experience. Post-Purchase Support:
Provide exceptional post-purchase support, such as easy returns, fast
complaint resolution, and follow-ups to ensure customer satisfaction.
Exclusive Offers: Provide exclusive deals and early access to sales for
loyal customers to make them feel valued.

\hypertarget{top-customers-by-sales}{%
\subsection{Top customers by sales}\label{top-customers-by-sales}}

\begin{Shaded}
\begin{Highlighting}[]
\CommentTok{\# Top 10 Customers by Sales}
\NormalTok{top\_customers\_by\_sales }\OtherTok{\textless{}{-}}\NormalTok{ data }\SpecialCharTok{\%\textgreater{}\%}
  \FunctionTok{group\_by}\NormalTok{(Customer\_Name) }\SpecialCharTok{\%\textgreater{}\%}
  \FunctionTok{summarise}\NormalTok{(}\AttributeTok{Total\_Sales =} \FunctionTok{sum}\NormalTok{(Sales, }\AttributeTok{na.rm =} \ConstantTok{TRUE}\NormalTok{)) }\SpecialCharTok{\%\textgreater{}\%}
  \FunctionTok{arrange}\NormalTok{(}\FunctionTok{desc}\NormalTok{(Total\_Sales)) }\SpecialCharTok{\%\textgreater{}\%}
  \FunctionTok{top\_n}\NormalTok{(}\DecValTok{10}\NormalTok{)}
\end{Highlighting}
\end{Shaded}

\begin{verbatim}
## Selecting by Total_Sales
\end{verbatim}

\begin{Shaded}
\begin{Highlighting}[]
\CommentTok{\# Plotting Top 10 Customers by Sales}
\FunctionTok{ggplot}\NormalTok{(top\_customers\_by\_sales, }\FunctionTok{aes}\NormalTok{(}\AttributeTok{x =} \FunctionTok{reorder}\NormalTok{(Customer\_Name, Total\_Sales), }\AttributeTok{y =}\NormalTok{ Total\_Sales)) }\SpecialCharTok{+}
  \FunctionTok{geom\_col}\NormalTok{(}\AttributeTok{fill =} \StringTok{"steelblue"}\NormalTok{, }\AttributeTok{width =} \FloatTok{0.6}\NormalTok{) }\SpecialCharTok{+}
  \FunctionTok{coord\_flip}\NormalTok{() }\SpecialCharTok{+}
  \FunctionTok{labs}\NormalTok{(}\AttributeTok{title =} \StringTok{"Top 10 Customers by Sales"}\NormalTok{, }\AttributeTok{x =} \StringTok{"Customer Name"}\NormalTok{, }\AttributeTok{y =} \StringTok{"Total Sales"}\NormalTok{) }\SpecialCharTok{+}
  \FunctionTok{theme\_minimal}\NormalTok{(}\AttributeTok{base\_size =} \DecValTok{14}\NormalTok{) }\SpecialCharTok{+}
  \FunctionTok{theme}\NormalTok{(}\AttributeTok{text =} \FunctionTok{element\_text}\NormalTok{(}\AttributeTok{family =} \StringTok{"serif"}\NormalTok{),}
        \AttributeTok{plot.title =} \FunctionTok{element\_text}\NormalTok{(}\AttributeTok{hjust =} \FloatTok{0.5}\NormalTok{, }\AttributeTok{size =} \DecValTok{16}\NormalTok{, }\AttributeTok{face =} \StringTok{"bold"}\NormalTok{),}
        \AttributeTok{axis.title.y =} \FunctionTok{element\_text}\NormalTok{(}\AttributeTok{face =} \StringTok{"bold"}\NormalTok{))}
\end{Highlighting}
\end{Shaded}

\includegraphics{Store-Sales_files/figure-latex/top_customers-1.pdf}

\hypertarget{top-profit-cities}{%
\subsection{Top profit cities}\label{top-profit-cities}}

\begin{Shaded}
\begin{Highlighting}[]
\CommentTok{\#Top 10 Cities by Profit and Sales}
\NormalTok{top\_cities\_by\_profit\_sales }\OtherTok{\textless{}{-}}\NormalTok{ data }\SpecialCharTok{\%\textgreater{}\%}
  \FunctionTok{group\_by}\NormalTok{(City) }\SpecialCharTok{\%\textgreater{}\%}
  \FunctionTok{summarise}\NormalTok{(}\AttributeTok{Total\_Sales =} \FunctionTok{sum}\NormalTok{(Sales, }\AttributeTok{na.rm =} \ConstantTok{TRUE}\NormalTok{), }\AttributeTok{Total\_Profit =} \FunctionTok{sum}\NormalTok{(Profit, }\AttributeTok{na.rm =} \ConstantTok{TRUE}\NormalTok{)) }\SpecialCharTok{\%\textgreater{}\%}
  \FunctionTok{arrange}\NormalTok{(}\FunctionTok{desc}\NormalTok{(Total\_Sales)) }\SpecialCharTok{\%\textgreater{}\%}
  \FunctionTok{top\_n}\NormalTok{(}\DecValTok{10}\NormalTok{)}
\end{Highlighting}
\end{Shaded}

\begin{verbatim}
## Selecting by Total_Profit
\end{verbatim}

\begin{Shaded}
\begin{Highlighting}[]
\CommentTok{\# Plotting Top 10 Cities by Profit and Sales}
\FunctionTok{ggplot}\NormalTok{(top\_cities\_by\_profit\_sales, }\FunctionTok{aes}\NormalTok{(}\AttributeTok{x =} \FunctionTok{reorder}\NormalTok{(City, Total\_Sales), }\AttributeTok{y =}\NormalTok{ Total\_Sales, }\AttributeTok{fill =}\NormalTok{ Total\_Profit)) }\SpecialCharTok{+}
  \FunctionTok{geom\_col}\NormalTok{() }\SpecialCharTok{+}
  \FunctionTok{coord\_flip}\NormalTok{() }\SpecialCharTok{+}
  \FunctionTok{labs}\NormalTok{(}\AttributeTok{title =} \StringTok{"Top 10 Cities by Profit and Sales"}\NormalTok{, }\AttributeTok{x =} \StringTok{"City"}\NormalTok{, }\AttributeTok{y =} \StringTok{"Total Sales"}\NormalTok{) }\SpecialCharTok{+}
  \FunctionTok{scale\_fill\_gradient}\NormalTok{(}\AttributeTok{low =} \StringTok{"yellow"}\NormalTok{, }\AttributeTok{high =} \StringTok{"red"}\NormalTok{, }\AttributeTok{name =} \StringTok{"Profit"}\NormalTok{) }\SpecialCharTok{+}
  \FunctionTok{theme\_minimal}\NormalTok{()}
\end{Highlighting}
\end{Shaded}

\includegraphics{Store-Sales_files/figure-latex/cities-1.pdf}

\hypertarget{top-profit-making-categories-and-sub-categories}{%
\subsection{Top profit making categories and
sub-categories}\label{top-profit-making-categories-and-sub-categories}}

\begin{Shaded}
\begin{Highlighting}[]
\CommentTok{\# Top 10 Profit{-}Making Categories and Sub{-}Categories}
\NormalTok{top\_profit\_categories\_subcategories }\OtherTok{\textless{}{-}}\NormalTok{ data }\SpecialCharTok{\%\textgreater{}\%}
  \FunctionTok{group\_by}\NormalTok{(Category, Sub\_Category) }\SpecialCharTok{\%\textgreater{}\%}
  \FunctionTok{summarise}\NormalTok{(}\AttributeTok{Total\_Profit =} \FunctionTok{sum}\NormalTok{(Profit, }\AttributeTok{na.rm =} \ConstantTok{TRUE}\NormalTok{)) }\SpecialCharTok{\%\textgreater{}\%}
  \FunctionTok{arrange}\NormalTok{(}\FunctionTok{desc}\NormalTok{(Total\_Profit)) }\SpecialCharTok{\%\textgreater{}\%}
  \FunctionTok{top\_n}\NormalTok{(}\DecValTok{10}\NormalTok{)}
\end{Highlighting}
\end{Shaded}

\begin{verbatim}
## `summarise()` has grouped output by 'Category'. You can override using the
## `.groups` argument.
## Selecting by Total_Profit
\end{verbatim}

\begin{Shaded}
\begin{Highlighting}[]
\CommentTok{\# Plotting Top 10 Profit{-}Making Categories and Sub{-}Categories}
\FunctionTok{ggplot}\NormalTok{(top\_profit\_categories\_subcategories, }\FunctionTok{aes}\NormalTok{(}\AttributeTok{x =} \FunctionTok{reorder}\NormalTok{(Sub\_Category, Total\_Profit), }\AttributeTok{y =}\NormalTok{ Total\_Profit, }\AttributeTok{fill =}\NormalTok{ Category)) }\SpecialCharTok{+}
  \FunctionTok{geom\_col}\NormalTok{(}\AttributeTok{width =} \FloatTok{0.6}\NormalTok{, }\AttributeTok{color =} \StringTok{"black"}\NormalTok{) }\SpecialCharTok{+}
  \FunctionTok{coord\_flip}\NormalTok{() }\SpecialCharTok{+}
  \FunctionTok{labs}\NormalTok{(}\AttributeTok{title =} \StringTok{"Top 10 Profit{-}Making Categories and Sub{-}Categories"}\NormalTok{, }\AttributeTok{x =} \StringTok{"Sub{-}Category"}\NormalTok{, }\AttributeTok{y =} \StringTok{"Total Profit"}\NormalTok{) }\SpecialCharTok{+}
  \FunctionTok{theme\_minimal}\NormalTok{(}\AttributeTok{base\_size =} \DecValTok{14}\NormalTok{) }\SpecialCharTok{+}
  \FunctionTok{theme}\NormalTok{(}\AttributeTok{text =} \FunctionTok{element\_text}\NormalTok{(}\AttributeTok{family =} \StringTok{"serif"}\NormalTok{),}
        \AttributeTok{plot.title =} \FunctionTok{element\_text}\NormalTok{(}\AttributeTok{hjust =} \FloatTok{0.5}\NormalTok{, }\AttributeTok{size =} \DecValTok{16}\NormalTok{, }\AttributeTok{face =} \StringTok{"bold"}\NormalTok{),}
        \AttributeTok{axis.title.y =} \FunctionTok{element\_text}\NormalTok{(}\AttributeTok{face =} \StringTok{"bold"}\NormalTok{)) }\SpecialCharTok{+}
  \FunctionTok{scale\_fill\_brewer}\NormalTok{(}\AttributeTok{palette =} \StringTok{"Set2"}\NormalTok{)}
\end{Highlighting}
\end{Shaded}

\includegraphics{Store-Sales_files/figure-latex/top_categories-1.pdf}

\hypertarget{category-wise-returns-by-the-customer}{%
\subsection{Category wise returns by the
customer}\label{category-wise-returns-by-the-customer}}

\begin{Shaded}
\begin{Highlighting}[]
\CommentTok{\# Number of Category{-}Wise Returns}
\NormalTok{category\_wise\_returns }\OtherTok{\textless{}{-}}\NormalTok{ data }\SpecialCharTok{\%\textgreater{}\%}
  \FunctionTok{filter}\NormalTok{(Returns }\SpecialCharTok{==} \StringTok{"Yes"}\NormalTok{) }\SpecialCharTok{\%\textgreater{}\%}
  \FunctionTok{group\_by}\NormalTok{(Category) }\SpecialCharTok{\%\textgreater{}\%}
  \FunctionTok{summarise}\NormalTok{(}\AttributeTok{Num\_Returns =} \FunctionTok{n}\NormalTok{())}

\CommentTok{\# Plotting Number of Category{-}Wise Returns}
\FunctionTok{ggplot}\NormalTok{(category\_wise\_returns, }\FunctionTok{aes}\NormalTok{(}\AttributeTok{x =} \FunctionTok{reorder}\NormalTok{(Category, Num\_Returns), }\AttributeTok{y =}\NormalTok{ Num\_Returns, }\AttributeTok{fill =}\NormalTok{ Category)) }\SpecialCharTok{+}
  \FunctionTok{geom\_col}\NormalTok{(}\AttributeTok{color =} \StringTok{"black"}\NormalTok{, }\AttributeTok{width =} \FloatTok{0.8}\NormalTok{) }\SpecialCharTok{+}
  \FunctionTok{labs}\NormalTok{(}\AttributeTok{title =} \StringTok{"Number of Category{-}Wise Returns"}\NormalTok{, }\AttributeTok{x =} \StringTok{"Category"}\NormalTok{, }\AttributeTok{y =} \StringTok{"Number of Returns"}\NormalTok{) }\SpecialCharTok{+}
  \FunctionTok{theme\_minimal}\NormalTok{(}\AttributeTok{base\_size =} \DecValTok{10}\NormalTok{) }\SpecialCharTok{+}
  \FunctionTok{theme}\NormalTok{(}\AttributeTok{text =} \FunctionTok{element\_text}\NormalTok{(}\AttributeTok{family =} \StringTok{"serif"}\NormalTok{),}
        \AttributeTok{plot.title =} \FunctionTok{element\_text}\NormalTok{(}\AttributeTok{hjust =} \FloatTok{0.5}\NormalTok{, }\AttributeTok{size =} \DecValTok{16}\NormalTok{, }\AttributeTok{face =} \StringTok{"bold"}\NormalTok{),}
        \AttributeTok{axis.title.y =} \FunctionTok{element\_text}\NormalTok{(}\AttributeTok{face =} \StringTok{"bold"}\NormalTok{)) }\SpecialCharTok{+}
  \FunctionTok{scale\_fill\_brewer}\NormalTok{(}\AttributeTok{palette =} \StringTok{"Set1"}\NormalTok{)}
\end{Highlighting}
\end{Shaded}

\includegraphics{Store-Sales_files/figure-latex/returns-1.pdf}

\hypertarget{conclusion}{%
\subsection{Conclusion}\label{conclusion}}

The dataset reveals crucial insights into sales performance across
various product categories, customer segments, payment modes, and
regions. By understanding these patterns, the company can take targeted
actions to boost revenue, such as optimizing inventory, tailoring
marketing strategies, and offering improved delivery options. While
certain regions and segments drive most of the sales, there are
opportunities for growth in underperforming areas. Additionally,
expanding popular product lines and enhancing the customer experience
through preferred payment methods and shipping options will help
maintain competitiveness. Strategic initiatives such as personalized
offers and loyalty programs can further increase customer retention and
loyalty, leading to sustainable business growth.

\end{document}
